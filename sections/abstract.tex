\begin{abstract}
In WoT environment, smart things provide RESTful services to expose their resources and operations. There are lots of smart things that offer the same functionalities but have different service interfaces. Because of the high coupling between device service instances and process specifications like BPEL, the cost of reusing a BPEL specification between different device environments is very high. We propose a device-adaptive service composition framework for WoT environment, in order to help users to apply the business process and service composition technologies more conveniently. In the framework, we design an activity description model, which is a semantic description for business activities, to overcome the shortcoming of directly binding the process and the service. Then, a matching mechanism between the model and the WADL of device services is proposed to select candidate services for the composition. Furthermore, we represent the matching result in a general service model, with which the source code of the general service can be automatically generated. The general service is a unified encapsulation for device services that match the functionalities of business activity. So user can interact with the general service instead of the origin services on the device, which decouple the process specification and the actual device services. A case study is offered to illustrate how to apply our framework in a intelligent charging pile sharing platform. 
\end{abstract}
\bigskip
\begin{IEEEkeywords}
WoT;service composition;
\end{IEEEkeywords}