\section{Related Work}
\label{Related Work}
To integrate traditional enterprise services and wireless sensor networks, \cite{guinard2009towards} proposes a lightweight approach following the success of Web 2.0 "mashups”, proposing and implementing a collection of RESTful services to expose sensor nodes to web resources so that all the nodes become part of a ‘Web of Thing’ and interact with existing nodes to compose their services. \cite{mayer2010facilitating} combines the above work with an AutoWot platform, which offers a generic way to model Web resources and build Web components in order to facilitate the integration of smart devices into web.The integration, however, is based on particular service instances for service composition, failing to describe the composition from a abstract level. 

\cite{ostermaier2010webplug} designs a WebPlug framework to strengthen the combination of real physical devices with virtual resources. In the framework, users can get access to physical resources by URL as well as the MetaURL defined by framework to obtain the related information about the device. For example adding “@history” after the URL of the device can archive the history information of this resource.SemSense proposed by \cite{moraru2011exposing} is composed of data collection module, storage module, semantic module and release module. Data are first collected from physical sensor, enriched by semantic annotation using LinkedData and finally published. Both approaches above are limited to annotating the resources, and are lack of the coordination among devices and services. 

For the semantic extension of process description language, traditional methods like BPEL have a low precision about the similarity between conceptual similar service, due to the strict matching strategy\cite{nitzsche2007ontology}. WSDL-S\cite{WSDLS} can't distinguish between service type(concept) and service instance(individual), as it's restriction in the expansion of the service concept. Likewise, in OWL-S\cite{OWLS}, there is no significant difference between the process instance (the actual combination of services) and conceptual process (only relating to the concept of services). 

Combining BPEL and RESTful service composition, \cite{pautasso2009restful} proposes an extension of the BPEL standard, which promotes the abstraction level of resources. Four basic HTTP methods (GET, PUT, POST, DELETE) are extended as a service invocation parameters in BPEL activity, so that the HTTP service can be directly used in the BPEL process. Similarly, \cite{zhang2014research} extends both HTTP request and BPEL, in order to composite asynchronous RESTful services in IoT environment. Researchers first extend the HTTP request header with asynchronous call related informations, and then add a HttpVariable tag as subtype of the Variable tag in BPEL. Thus the HTTP Code, HTTP headers and content of the HTTP Body is attached to the service description as a Variable, and the four basic HTTP methods are added to the BPEL activity description. 

In \cite{rauf2010modeling}, researchers propose a RESTful service composition modeling method based on UML, using a conceptual resource model to describe the static composition structure. Besides, a state machine is used to represent the behavior information of the HTTP methods like PUT and POST. But the drawback is that it is hard to implement the actual service composition based on the model. 

\cite{rathod2015towards} proposes an approach termed as Web Service Resource Bundle(WSRB), considering the dependencies between the candidate services and binding those services during the invocation from the client, which offers a novel runtime compositing concept. But WSRB focuses on the fixed and predefined composition, failing to support the dynamic composition based on the business requirement. 

The above literature survey shows that it is hard to directly apply existing business process and service composition technologies to WoT. Compared with approaches mentioned in those survey, our framework pays more attention to the WoT characteristics that there are lots of devices with similar functionalities and the device replacement frequency is high. 